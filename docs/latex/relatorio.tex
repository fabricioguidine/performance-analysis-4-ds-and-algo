% Relatório Técnico - Análise de Estruturas de Dados
% Análise de Algoritmos de Ordenação, Tabelas Hash e Árvores Balanceadas
% Dataset: Book Depository

\documentclass[
    12pt,
    a4paper,
    openright,
    oneside,
    english,
    brazil,
    sumario=tradicional
]{abntex2}

% Packages
\usepackage[utf8]{inputenc}
\usepackage[T1]{fontenc}
\usepackage{graphicx}
\usepackage{booktabs}
\usepackage{longtable}
\usepackage{listings}
\usepackage{xcolor}
\usepackage{url}
\usepackage{hyperref}
\usepackage{amsmath}
\usepackage{float}
\usepackage{caption}
\usepackage{subcaption}
\usepackage{pgfplots}
\usepackage{pgfplotstable}

% Configuration
\definecolor{codegreen}{rgb}{0,0.6,0}
\definecolor{codegray}{rgb}{0.5,0.5,0.5}
\definecolor{codepurple}{rgb}{0.58,0,0.82}
\definecolor{backcolour}{rgb}{0.95,0.95,0.92}

\lstdefinestyle{mystyle}{
    backgroundcolor=\color{backcolour},
    commentstyle=\color{codegreen},
    keywordstyle=\color{magenta},
    numberstyle=\tiny\color{codegray},
    stringstyle=\color{codepurple},
    basicstyle=\ttfamily\footnotesize,
    breakatwhitespace=false,
    breaklines=true,
    captionpos=b,
    keepspaces=true,
    numbers=left,
    numbersep=5pt,
    showspaces=false,
    showstringspaces=false,
    showtabs=false,
    tabsize=2
}

\lstset{style=mystyle}

% PDF Metadata
\hypersetup{
    pdftitle={Análise de Estruturas de Dados - Book Depository Dataset},
    pdfauthor={Equipe do Projeto},
    pdfsubject={Análise de Algoritmos e Estruturas de Dados},
    pdfkeywords={algoritmos, estruturas de dados, ordenação, hash tables, árvores balanceadas},
    colorlinks=true,
    linkcolor=blue,
    filecolor=magenta,
    urlcolor=cyan,
}

% Document Information
\titulo{Análise de Estruturas de Dados e Algoritmos\\
Utilizando o Dataset Book Depository}
\autor{Débora Duarte\\Fabrício Guidine\\Walkíria Garcia}
\local{Juiz de Fora}
\data{\today}
\instituicao{%
Universidade Federal de Juiz de Fora
\par
Departamento de Ciência da Computação
\par
DCC012 - Estruturas de Dados II
}
\tipotrabalho{Trabalho de Disciplina}

% Index
\makeindex

\begin{document}

% Front Matter
\pretextual
\imprimircapa
\imprimirfolhaderosto

\begin{resumo}
Este trabalho apresenta uma análise comparativa de algoritmos de ordenação, implementação de tabelas hash e avaliação de estruturas de dados balanceadas utilizando o dataset Book Depository. Os experimentos avaliam o desempenho dos algoritmos QuickSort e HeapSort, implementam tabelas hash para identificar autores mais frequentes e comparam o desempenho de árvores Vermelho-Preto e B+ em operações de inserção e busca. Os resultados demonstram as características de cada estrutura de dados e algoritmo, fornecendo insights sobre sua aplicabilidade em diferentes cenários.
\end{resumo}

\palavraschave{algoritmos de ordenação, tabelas hash, árvores balanceadas, análise de desempenho, estruturas de dados}

\listoffigures
\listoftables
\sumario

% Main Content
\textual

\chapter{Introdução}

Este relatório apresenta uma análise completa de estruturas de dados e algoritmos fundamentais em Ciência da Computação, utilizando o dataset Book Depository como base de dados para os experimentos realizados.

\section{Contexto}

O projeto foi desenvolvido como parte dos requisitos da disciplina DCC012 - Estruturas de Dados II, do Departamento de Ciência da Computação da Universidade Federal de Juiz de Fora, no semestre 2020.1.

\section{Objetivos}

Os principais objetivos deste trabalho incluem:

\begin{itemize}
    \item Implementar e analisar algoritmos de ordenação (QuickSort e HeapSort)
    \item Implementar tabelas hash para análise de frequência de autores
    \item Avaliar o desempenho de estruturas de dados balanceadas (Árvore Vermelho-Preto e Árvore B+)
    \item Comparar métricas de desempenho entre diferentes estruturas e algoritmos
    \item Gerar resultados empíricos através de experimentos controlados
\end{itemize}

\section{Organização do Relatório}

Este relatório está organizado da seguinte forma:

\begin{itemize}
    \item \textbf{Capítulo 2}: Revisão bibliográfica sobre as estruturas e algoritmos estudados
    \item \textbf{Capítulo 3}: Metodologia e descrição dos experimentos
    \item \textbf{Capítulo 4}: Análise de algoritmos de ordenação
    \item \textbf{Capítulo 5}: Análise de tabelas hash
    \item \textbf{Capítulo 6}: Análise de estruturas de dados balanceadas
    \item \textbf{Capítulo 7}: Resultados e discussão
    \item \textbf{Capítulo 8}: Conclusões e trabalhos futuros
\end{itemize}

\chapter{Revisão Bibliográfica}

\section{Algoritmos de Ordenação}

\subsection{QuickSort}

O QuickSort é um algoritmo de ordenação eficiente que utiliza a estratégia de divisão e conquista...

\subsection{HeapSort}

O HeapSort utiliza a propriedade de heap para ordenar elementos...

\section{Tabelas Hash}

As tabelas hash são estruturas de dados que permitem acesso rápido a elementos através de funções de hash...

\section{Estruturas de Dados Balanceadas}

\subsection{Árvore Vermelho-Preto}

Árvores Vermelho-Preto são árvores binárias de busca auto-equilibradas...

\subsection{Árvore B+}

Árvores B+ são estruturas de dados otimizadas para armazenamento em disco...

\chapter{Metodologia}

\section{Dataset}

O dataset Book Depository foi obtido do Kaggle e contém informações sobre livros, incluindo:

\begin{itemize}
    \item Dados de autores e suas obras
    \item Informações sobre livros (título, ISBN, categorias)
    \item Rankings de bestsellers
    \item Avaliações e classificações
\end{itemize}

\section{Métricas de Desempenho}

Para cada algoritmo e estrutura de dados, foram medidos:

\begin{itemize}
    \item \textbf{Número de comparações}: Quantidade de comparações entre elementos
    \item \textbf{Número de trocas/cópias}: Quantidade de operações de movimentação de dados
    \item \textbf{Tempo de execução}: Tempo total de processamento em milissegundos
\end{itemize}

\section{Ambiente de Execução}

Os experimentos foram executados em:
\begin{itemize}
    \item Linguagem: Java
    \item Ambiente: JVM (Java Virtual Machine)
    \item Sistema operacional: Windows/Linux
    \item Métricas: Tempo de máquina (não tempo de relógio)
\end{itemize}

\chapter{Análise de Algoritmos de Ordenação}

\section{Metodologia dos Experimentos}

Os experimentos de ordenação foram realizados com diferentes tamanhos de entrada...

\section{Resultados}

% Placeholder for results from tests/sorting/

\subsection{QuickSort}

% Include results from tests/sorting/quicksort_results.txt

\subsection{HeapSort}

% Include results from tests/sorting/heapsort_results.txt

\section{Análise Comparativa}

\chapter{Análise de Tabelas Hash}

\section{Implementação}

As tabelas hash foram implementadas utilizando endereçamento aberto com double hashing...

\section{Resultados}

% Placeholder for results from tests/hashtable/

\section{Análise de Autores Mais Frequentes}

% Include results from tests/hashtable/most_frequent_authors.txt

\chapter{Análise de Estruturas de Dados Balanceadas}

\section{Árvore Vermelho-Preto}

% Placeholder for results from tests/trees/redblack_tree/

\section{Árvore B+ (d=2)}

% Placeholder for results from tests/trees/bplustree_d2/

\section{Árvore B+ (d=20)}

% Placeholder for results from tests/trees/bplustree_d20/

\section{Análise Comparativa}

% Include comparison tables and graphs

\chapter{Resultados e Discussão}

\section{Síntese dos Resultados}

\section{Interpretação dos Dados}

\section{Limitações dos Experimentos}

\chapter{Conclusões e Trabalhos Futuros}

\section{Conclusões}

Este trabalho apresentou uma análise abrangente de diferentes estruturas de dados e algoritmos...

\section{Trabalhos Futuros}

Possíveis extensões deste trabalho incluem:

\begin{itemize}
    \item Implementação de outras estruturas de dados balanceadas
    \item Análise de complexidade espacial
    \item Otimizações adicionais dos algoritmos
    \item Análise em datasets maiores
\end{itemize}

% Back Matter
\postextual

\bibliography{referencias}

% Appendix
\begin{apendicesenv}

\chapter{Resultados Completos}

% Include full results tables here

\chapter{Código-Fonte}

% Include relevant code snippets if needed

\end{apendicesenv}

\end{document}

